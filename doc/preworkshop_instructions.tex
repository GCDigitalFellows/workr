\documentclass[]{article}
\usepackage{lmodern}
\usepackage{amssymb,amsmath}
\usepackage{ifxetex,ifluatex}
\usepackage{fixltx2e} % provides \textsubscript
\ifnum 0\ifxetex 1\fi\ifluatex 1\fi=0 % if pdftex
  \usepackage[T1]{fontenc}
  \usepackage[utf8]{inputenc}
\else % if luatex or xelatex
  \ifxetex
    \usepackage{mathspec}
  \else
    \usepackage{fontspec}
  \fi
  \defaultfontfeatures{Ligatures=TeX,Scale=MatchLowercase}
\fi
% use upquote if available, for straight quotes in verbatim environments
\IfFileExists{upquote.sty}{\usepackage{upquote}}{}
% use microtype if available
\IfFileExists{microtype.sty}{%
\usepackage{microtype}
\UseMicrotypeSet[protrusion]{basicmath} % disable protrusion for tt fonts
}{}
\usepackage[margin=1in]{geometry}
\usepackage{hyperref}
\hypersetup{unicode=true,
            pdftitle={Pre-workshop Installation Instructions},
            pdfborder={0 0 0},
            breaklinks=true}
\urlstyle{same}  % don't use monospace font for urls
\usepackage{color}
\usepackage{fancyvrb}
\newcommand{\VerbBar}{|}
\newcommand{\VERB}{\Verb[commandchars=\\\{\}]}
\DefineVerbatimEnvironment{Highlighting}{Verbatim}{commandchars=\\\{\}}
% Add ',fontsize=\small' for more characters per line
\usepackage{framed}
\definecolor{shadecolor}{RGB}{248,248,248}
\newenvironment{Shaded}{\begin{snugshade}}{\end{snugshade}}
\newcommand{\AlertTok}[1]{\textcolor[rgb]{0.94,0.16,0.16}{#1}}
\newcommand{\AnnotationTok}[1]{\textcolor[rgb]{0.56,0.35,0.01}{\textbf{\textit{#1}}}}
\newcommand{\AttributeTok}[1]{\textcolor[rgb]{0.77,0.63,0.00}{#1}}
\newcommand{\BaseNTok}[1]{\textcolor[rgb]{0.00,0.00,0.81}{#1}}
\newcommand{\BuiltInTok}[1]{#1}
\newcommand{\CharTok}[1]{\textcolor[rgb]{0.31,0.60,0.02}{#1}}
\newcommand{\CommentTok}[1]{\textcolor[rgb]{0.56,0.35,0.01}{\textit{#1}}}
\newcommand{\CommentVarTok}[1]{\textcolor[rgb]{0.56,0.35,0.01}{\textbf{\textit{#1}}}}
\newcommand{\ConstantTok}[1]{\textcolor[rgb]{0.00,0.00,0.00}{#1}}
\newcommand{\ControlFlowTok}[1]{\textcolor[rgb]{0.13,0.29,0.53}{\textbf{#1}}}
\newcommand{\DataTypeTok}[1]{\textcolor[rgb]{0.13,0.29,0.53}{#1}}
\newcommand{\DecValTok}[1]{\textcolor[rgb]{0.00,0.00,0.81}{#1}}
\newcommand{\DocumentationTok}[1]{\textcolor[rgb]{0.56,0.35,0.01}{\textbf{\textit{#1}}}}
\newcommand{\ErrorTok}[1]{\textcolor[rgb]{0.64,0.00,0.00}{\textbf{#1}}}
\newcommand{\ExtensionTok}[1]{#1}
\newcommand{\FloatTok}[1]{\textcolor[rgb]{0.00,0.00,0.81}{#1}}
\newcommand{\FunctionTok}[1]{\textcolor[rgb]{0.00,0.00,0.00}{#1}}
\newcommand{\ImportTok}[1]{#1}
\newcommand{\InformationTok}[1]{\textcolor[rgb]{0.56,0.35,0.01}{\textbf{\textit{#1}}}}
\newcommand{\KeywordTok}[1]{\textcolor[rgb]{0.13,0.29,0.53}{\textbf{#1}}}
\newcommand{\NormalTok}[1]{#1}
\newcommand{\OperatorTok}[1]{\textcolor[rgb]{0.81,0.36,0.00}{\textbf{#1}}}
\newcommand{\OtherTok}[1]{\textcolor[rgb]{0.56,0.35,0.01}{#1}}
\newcommand{\PreprocessorTok}[1]{\textcolor[rgb]{0.56,0.35,0.01}{\textit{#1}}}
\newcommand{\RegionMarkerTok}[1]{#1}
\newcommand{\SpecialCharTok}[1]{\textcolor[rgb]{0.00,0.00,0.00}{#1}}
\newcommand{\SpecialStringTok}[1]{\textcolor[rgb]{0.31,0.60,0.02}{#1}}
\newcommand{\StringTok}[1]{\textcolor[rgb]{0.31,0.60,0.02}{#1}}
\newcommand{\VariableTok}[1]{\textcolor[rgb]{0.00,0.00,0.00}{#1}}
\newcommand{\VerbatimStringTok}[1]{\textcolor[rgb]{0.31,0.60,0.02}{#1}}
\newcommand{\WarningTok}[1]{\textcolor[rgb]{0.56,0.35,0.01}{\textbf{\textit{#1}}}}
\usepackage{graphicx,grffile}
\makeatletter
\def\maxwidth{\ifdim\Gin@nat@width>\linewidth\linewidth\else\Gin@nat@width\fi}
\def\maxheight{\ifdim\Gin@nat@height>\textheight\textheight\else\Gin@nat@height\fi}
\makeatother
% Scale images if necessary, so that they will not overflow the page
% margins by default, and it is still possible to overwrite the defaults
% using explicit options in \includegraphics[width, height, ...]{}
\setkeys{Gin}{width=\maxwidth,height=\maxheight,keepaspectratio}
\IfFileExists{parskip.sty}{%
\usepackage{parskip}
}{% else
\setlength{\parindent}{0pt}
\setlength{\parskip}{6pt plus 2pt minus 1pt}
}
\setlength{\emergencystretch}{3em}  % prevent overfull lines
\providecommand{\tightlist}{%
  \setlength{\itemsep}{0pt}\setlength{\parskip}{0pt}}
\setcounter{secnumdepth}{0}
% Redefines (sub)paragraphs to behave more like sections
\ifx\paragraph\undefined\else
\let\oldparagraph\paragraph
\renewcommand{\paragraph}[1]{\oldparagraph{#1}\mbox{}}
\fi
\ifx\subparagraph\undefined\else
\let\oldsubparagraph\subparagraph
\renewcommand{\subparagraph}[1]{\oldsubparagraph{#1}\mbox{}}
\fi

%%% Use protect on footnotes to avoid problems with footnotes in titles
\let\rmarkdownfootnote\footnote%
\def\footnote{\protect\rmarkdownfootnote}

%%% Change title format to be more compact
\usepackage{titling}

% Create subtitle command for use in maketitle
\providecommand{\subtitle}[1]{
  \posttitle{
    \begin{center}\large#1\end{center}
    }
}

\setlength{\droptitle}{-2em}

  \title{Pre-workshop Installation Instructions}
    \pretitle{\vspace{\droptitle}\centering\huge}
  \posttitle{\par}
    \author{}
    \preauthor{}\postauthor{}
    \date{}
    \predate{}\postdate{}
  

\begin{document}
\maketitle

\hypertarget{installupdate-r-rstudio}{%
\subsection{Install/Update R \& RStudio}\label{installupdate-r-rstudio}}

Contents borrowed and modified from
\href{https://uvastatlab.github.io/phdplus/intror.html}{UVA's Data
Science Essentials in R series}

\hypertarget{before-the-first-session}{%
\subsubsection{Before the first
session}\label{before-the-first-session}}

To participate in the R workshop, please bring a laptop with R and
RStudio installed. We recommend that you have the latest version of R
(3.6.1), the \texttt{tidyverse} package (1.2.1), and the \texttt{learnr}
package (0.9.2.1). You need to have RStudio installed, but it is less
crucial that you are using the most recent version (1.2.5001).

Do you already have R and RStudio installed?

\begin{itemize}
\tightlist
\item
  No - follow the instructions for ``I do not have R installed''\\
\item
  Yes - follow the instructions for ``I have R installed''
\end{itemize}

\hypertarget{i-do-not-have-r-installed}{%
\subsubsection{``I do not have R
installed''}\label{i-do-not-have-r-installed}}

You should install R, RStudio, \texttt{tidyverse}, and \texttt{learnr}.

\hypertarget{installing-r}{%
\paragraph{Installing R}\label{installing-r}}

\hypertarget{windows}{%
\subparagraph{Windows:}\label{windows}}

\begin{enumerate}
\def\labelenumi{\arabic{enumi}.}
\tightlist
\item
  Go to \url{https://cloud.r-project.org/bin/windows/base/}
\item
  Click the ``Download R 3.6.1 for Windows'' link.
\item
  When the file finishes downloading, double-click to install. You
  should be able to click ``Next'' to all dialogs to finish the
  installation.
\end{enumerate}

\hypertarget{mac}{%
\subparagraph{Mac:}\label{mac}}

\begin{enumerate}
\def\labelenumi{\arabic{enumi}.}
\tightlist
\item
  Go to \url{https://cloud.r-project.org/bin/macosx/}
\item
  Click the link ``R-3.6.1.pkg''
\item
  When the file finishes downloading, double-click to install. You
  should be able to click ``Next'' to all dialogs to finish the
  installation.
\end{enumerate}

\hypertarget{linux}{%
\subparagraph{Linux:}\label{linux}}

For any adventurous Linux users in our group follow
\href{https://github.com/duckmayr/install-update-r-on-linux}{this guide}
to install/upgrade to the most recent version of R on Ubuntu (18.04) or
Mint (19).

\hypertarget{installing-rstudio}{%
\paragraph{Installing RStudio}\label{installing-rstudio}}

\begin{enumerate}
\def\labelenumi{\arabic{enumi}.}
\tightlist
\item
  Go to
  \href{https://www.rstudio.com/products/rstudio/download/\#download}{the
  RStudio download page}.
\item
  Under ``Installers for Supported Platforms'' select the appropriate
  installer for your operating system
\item
  When the file finishes downloading, double-click to install. You
  should be able to click ``Next'' to all dialogs to finish the
  installation.
\end{enumerate}

\hypertarget{installing-tidyverse}{%
\paragraph{\texorpdfstring{Installing
\texttt{tidyverse}}{Installing tidyverse}}\label{installing-tidyverse}}

\begin{enumerate}
\def\labelenumi{\arabic{enumi}.}
\tightlist
\item
  Open RStudio
\item
  Go to \texttt{Tools} \textgreater{} \texttt{Install\ Packages}
\item
  Enter \texttt{tidyverse}
\item
  Select \texttt{Install}
\end{enumerate}

Follow the same protocol for installing the \texttt{learnr} package, but
replace \texttt{tidyverse} with \texttt{learnr}.

\hypertarget{i-have-r-installed}{%
\subsubsection{``I have R installed''}\label{i-have-r-installed}}

The workshops run more smoothly when everyone is using the same version
of R, \texttt{tidyverse}, and \texttt{learnr}. Please update R,
\texttt{tidyverse}, and \texttt{learnr} if necessary (and less
crucially, RStudio).

\hypertarget{verify-r-version}{%
\paragraph{Verify R version}\label{verify-r-version}}

Open RStudio. At the top of the Console you will see session info. The
first line tells you which version of R you are using. If RStudio is
already open and you're deep in a session, type
\texttt{R.version.string} in the console and enter to print out the R
version.

Do you have R version 3.6.1 installed?

\begin{itemize}
\tightlist
\item
  No - follow the instructions for ``Updating R''
\item
  Yes - Great! Do you have \texttt{tidyverse} and \texttt{learnr}
  installed?

  \begin{itemize}
  \tightlist
  \item
    No or I don't know - See ``Installing \texttt{tidyverse}''
  \item
    Yes - Great! Go to Go to Tools \textgreater{} Check for Package
    Updates. If there's an update available for \texttt{tidyverse},
    install it.
  \end{itemize}
\end{itemize}

\hypertarget{updating-rrstudiotidyverselearnr}{%
\paragraph{Updating
R/RStudio/Tidyverse/learnr}\label{updating-rrstudiotidyverselearnr}}

\hypertarget{windows-1}{%
\subparagraph{Windows}\label{windows-1}}

To update R on Windows, try using the package \texttt{installr} (only
for Windows).

\begin{enumerate}
\def\labelenumi{\arabic{enumi}.}
\tightlist
\item
  Install and load installr:
\end{enumerate}

\begin{Shaded}
\begin{Highlighting}[]
\KeywordTok{install.packages}\NormalTok{(}\StringTok{"installr"}\NormalTok{)}
\KeywordTok{library}\NormalTok{(installr)}
\end{Highlighting}
\end{Shaded}

\begin{enumerate}
\def\labelenumi{\arabic{enumi}.}
\setcounter{enumi}{1}
\tightlist
\item
  Call \texttt{updateR()} function. This will start the updating process
  of your R installation by: ``finding the latest R version, downloading
  it, running the installer, deleting the installation file, copy and
  updating old packages to the new R installation.''\\
\item
  From within RStudio, go to Help \textgreater{} Check for Updates to
  install newer version of RStudio (if available, optional).
\end{enumerate}

\hypertarget{mac-1}{%
\subparagraph{Mac}\label{mac-1}}

On Mac, you can simply download and install the newest version of R.
When you restart RStudio, it will use the updated version of R.

\begin{enumerate}
\def\labelenumi{\arabic{enumi}.}
\tightlist
\item
  Go to \url{https://cloud.r-project.org/bin/macosx/}\\
\item
  Click the link ``R-3.6.1.pkg''\\
\item
  When the file finishes downloading, double-click to install. You
  should be able to click ``Next'' to all dialogs to finish the
  installation.\\
\item
  From within RStudio, go to Help \textgreater{} Check for Updates to
  install newer version of RStudio (if available, optional).\\
\item
  To update packages, go to Tools \textgreater{} Check for Package
  Updates. If updates are available, select All (or just
  \texttt{tidyverse}/\texttt{learnr}), and click Install Updates.
\end{enumerate}

\hypertarget{linux-1}{%
\subparagraph{Linux:}\label{linux-1}}

Again, for any adventurous Linux users in our group follow
\href{https://github.com/duckmayr/install-update-r-on-linux}{this guide}
to install/upgrade to the most recent version of R on Ubuntu (18.04) or
Mint (19).


\end{document}
